%%%%%%%%%%%%%%%%%%%%%%%%%%%%%%%%%%%%%%%%%%%%%%%%%%%%%%%%%%%%%%%%%%%%%%%%%%%%%%%%%%
\begin{frame}[fragile]\frametitle{}
\begin{center}
{\Large Syntax}
\end{center}

% \tiny{ (Ref: https://automatetheboringstuff.com/ )}
\end{frame}
%%%%%%%%%%%%%%%%%%%%%%%%%%%%%%%%%%%%%%%%%%%%%%%%%%%%%%%%%%%%%%%%%%%%%%%%%%%%%%%%%%%
\begin{frame}[fragile]  \frametitle{What is Python?}
\begin{itemize}
\item Python is an interpreted, object-oriented, high-level programming language with dynamic semantics.
\item Python is simple and easy to learn.
\item Python is open source, free and cross-platform.
\item Python provides high-level built in data structures.
\item Python is useful for rapid application development.
\item Python can be used as a scripting or glue language.
\item Python emphasizes readability.
\item Python supports modules and packages.
\item Python bugs or bad inputs will never cause a segmentation fault.
\end{itemize}
\end{frame}

%%%%%%%%%%%%%%%%%%%%%%%%%%%%%%%%%%%%%%%%%%%%%%%%%%%%%%%%%%%%%%%%%%%%%%%%%%%%%%%%%%%
\begin{frame}[fragile]  \frametitle{The Python shell, I}
  \begin{itemize}
  \item Python can be run from ``shell'', IDE, Notebook
 \item Shell/Command Line:
\begin{lstlisting}
> python

Python 3.5.3 | packaged by conda-forge | (default, May 12 2017, 16:16:49) [MSC v.1900 64 bit (AMD64)] on win32
Type "help", "copyright", "credits" or "license" for more information.
>>>
\end{lstlisting}
\item Start writing commands/expressions at the $>>>$ prompt.
\end{itemize}
\end{frame}

%%%%%%%%%%%%%%%%%%%%%%%%%%%%%%%%%%%%%%%%%%%%%%%%%%%%%%%%%%%%%%%%%%%%%%%%%%%%%%%%%%%
\begin{frame}[fragile]\frametitle{The Python shell, II}
\begin{itemize}
\item Expressions are evaluated and the result is printed:
	\begin{lstlisting}
	>>> 2+2
	4
	\end{lstlisting}

\item Line continuation with \\ 
\begin{lstlisting}
>>> "hello" + \
... " world!"
'hello world!'
\end{lstlisting}
\item The prompt changes to `\texttt{...}' on continuation lines and for loops, function definitions, etc.
\end{itemize}
\end{frame}

%%%%%%%%%%%%%%%%%%%%%%%%%%%%%%%%%%%%%%%%%%%%%%%%%%%%%%%%%%%%%%%%%%%%%%%%%%%%%%%%%%%
\begin{frame}[fragile]  \frametitle{Overall Syntax}
\begin{itemize}
\item Comments are indicated with ``\#''
\item Multiple statements on the same line are separated with ``;''
\item No semicolon at the end of lines.
%\item a long line continue on next with ``\'' (it is not always needed)
\item Scope is obtained through indentation. 
\item Always indent next line if ``:'' is at the end of current line.
\item One script is can be run or imported by other modules.
%\item  assignment uses the equal sign ``=''
\end{itemize}
\end{frame}


%%%%%%%%%%%%%%%%%%%%%%%%%%%%%%%%%%%%%%%%%%%%%%%%%%%%%%%%%%%%%%%%%%%%%%%%%%%%%%%%%%%
\begin{frame}[fragile]  \frametitle{Assignment}
\begin{itemize}
\item Assignment creates references, not values:
\lstinline{tmp = "hello"; tmp = 10}
the first string will be deallocated
%\item Contrary to C, assignment do not have value:
%\lstinline{y = (x = x + 1)}
% is invalid
\item As in C programming: \lstinline{x += 1} is valid
\item Pre/post increment/decrements: \lstinline{x++; ++x; x--;--x} are invalid
\item Multiple assignment (references to a unique object):\lstinline{x=y=z=1}
\item Multiple assignments: \lstinline{(x,y,z)=(3.5,5.5,`string')}
\item  Example of swapping variables value: \lstinline{(x,y)=(y,x)}
\end{itemize}
\end{frame}

%%%%%%%%%%%%%%%%%%%%%%%%%%%%%%%%%%%%%%%%%%%%%%%%%%%%%%%%%%%%%%%%%%%%%%%%%%%%%%%%%%%%
%\begin{frame}[fragile]  \frametitle{Special variables}
%\begin{itemize}
%\item Python relies on many special variables that can be accessed by your code.
%\item One is the \lstinline{name} variables.
%\item When a module is run, it contains the string \lstinline{main}.
%\item When the module is imported, it contains the modules name.
%\item You can add code that runs only when a module is called directly:\lstinline{if __name__ == __ main__: test()}
%\end{itemize}
%\end{frame}

%%%%%%%%%%%%%%%%%%%%%%%%%%%%%%%%%%%%%%%%%%%%%%%%%%%%%%%%%%%%%%%%%%%%%%%%%%%%%%%%%%%
\begin{frame}[fragile]\frametitle{Built-in object types}
  \begin{itemize}
  \item Numbers : \lstinline{3.1415, 1234, 999L, 3+4j}
  \item Strings : \lstinline{`spam', ``guido's''}
  \item Lists : \lstinline{[1, [2, `three'], 4]}
   \item Dictionaries :\lstinline|{`food':`spam', `taste':`yum'}|
  \item Tuples : \lstinline{(1,`spam', 4, `U')}
  \item Sets: \lstinline|{1,2,3,'foo','bar'}|
%  \item Files : \lstinline{text = open(`eggs', `r').read()}
  \end{itemize}
\end{frame}

%%%%%%%%%%%%%%%%%%%%%%%%%%%%%%%%%%%%%%%%%%%%%%%%%%%%%%%%%%%%%%%%%%%%%%%%%%%%%%%%%%%
\begin{frame}[fragile]\frametitle{Numbers}
  \begin{itemize}
  \item Integers : \lstinline{1234, -24, 0}
  \item Unlimited precision integers : \lstinline{ 999999999999L}
  \item Float : \lstinline{3.1415, 2.7122}
   \item Oct and hex :\lstinline|0177, 0x9ff|
  \item Complex : \lstinline{3+4j, 3.0+4.0j, 3J}
  \end{itemize}
\end{frame}

%%%%%%%%%%%%%%%%%%%%%%%%%%%%%%%%%%%%%%%%%%%%%%%%%%%%%%%%%%%%%%%%%%%%%%%%%%%%%%%%%%%
\begin{frame}[fragile]\frametitle{Strings (immutable sequences)}
  \begin{itemize}
  \item single quote \lstinline{s1 = `egg'}
\item double quotes \lstinline{s2 = ``spam's''}
\item triple quotes \lstinline{block = ```...'''}
\item concatenate \lstinline{s1 + s2}
\item repeat \lstinline{s2 * 3}
\item index,slice \lstinline{s2[i], s2[i:j]}
\item length \lstinline{len(s2)}
\item formatting \lstinline|``a {} parrot''.format(`dead')|
\item iteration \lstinline{for x in s2 # x loop through each character of s2}
\item membership \lstinline{`m' in s2}
  \end{itemize}
\end{frame}

%%%%%%%%%%%%%%%%%%%%%%%%%%%%%%%%%%%%%%%%%%%%%%%%%%%%%%%%%%%%%%%%%%%%%%%%%%%%%%%%%%%
\begin{frame}[fragile]\frametitle{Lists}
  \begin{itemize}
  \item Ordered collections of arbitrary objects
  \item Accessed by offset
  \item Variable length, heterogeneous, arbitrarily nest-able
  \item Mutable sequence
  \item Arrays of object references
  \end{itemize}
\end{frame}

%%%%%%%%%%%%%%%%%%%%%%%%%%%%%%%%%%%%%%%%%%%%%%%%%%%%%%%%%%%%%%%%%%%%%%%%%%%%%%%%%%%
\begin{frame}[fragile]\frametitle{Lists operations}
  \begin{itemize}
  \item empty list \lstinline{L = []}
    % \item create list \lstinline{L1 = range(4)}
  \item four items \lstinline{L2 = [0, 1, 2, 3]}
  \item nested \lstinline{L3 = ['abc', ['def', 'ghi']]}
  \item index \lstinline{L2[i], L3[i][j]}
  \item slice  \lstinline{L2[i:j]}, length \lstinline{len(L2)}
  \item concatenate \lstinline{L1 + L2}, repeat \lstinline{L2 * 3}
  \item iteration \lstinline{for x in L2}, membership \lstinline{3 in L2}
  \item methods \lstinline{L2.append(4), L2.sort(), L2.index(1), L2.reverse()}
  \item shrinking \lstinline{del L2[k], L2[i:j] = []}
  \item assignment \lstinline{L2[i] = 1, L2[i:j] = [4,5,6]}

  \end{itemize}
\end{frame}

%%%%%%%%%%%%%%%%%%%%%%%%%%%%%%%%%%%%%%%%%%%%%%%%%%%%%%%%%%%%%%%%%%%%%%%%%%%%%%%%%%%
\begin{frame}[fragile]\frametitle{Dictionaries}
  \begin{itemize}
  \item Accessed by key, not offset
  \item Unordered collections of arbitrary objects
  \item Variable length, heterogeneous, arbitrarily nest-able
  \item Of the category mutable mapping
  \item Tables of object references (hash tables)
  \end{itemize}
\end{frame}


%%%%%%%%%%%%%%%%%%%%%%%%%%%%%%%%%%%%%%%%%%%%%%%%%%%%%%%%%%%%%%%%%%%%%%%%%%%%%%%%%%%
\begin{frame}[fragile]\frametitle{Dictionaries operations}
  \begin{itemize}
  \item empty  \lstinline|d1 = {}|
  \item two-item \lstinline|d2 = {'spam': 2, 'eggs': 3}|
  \item nesting \lstinline|d3 = {'food': {'ham': 1, 'egg': 2}}|
  \item indexing \lstinline|d2['eggs'], d3['food']['ham']|
  \item methods \lstinline|d2.keys(), d2.values()|
  \item length \lstinline| len(d1)|
  \item add/change \lstinline|d2[key] = new|
  \item deleting \lstinline|del d2[key]|
  \end{itemize}
\end{frame}

%%%%%%%%%%%%%%%%%%%%%%%%%%%%%%%%%%%%%%%%%%%%%%%%%%%%%%%%%%%%%%%%%%%%%%%%%%%%%%%%%%%
\begin{frame}[fragile]\frametitle{tuples}
  \begin{itemize}
  \item They are like lists but immutable. Why Lists and Tuples?
  \item When you want to make sure the content won't change.
  \end{itemize}
\end{frame}

%%%%%%%%%%%%%%%%%%%%%%%%%%%%%%%%%%%%%%%%%%%%%%%%%%%%%%%%%%%%%%%%%%%%%%%%%%%%%%%%%%%
\begin{frame}[fragile]\frametitle{Files}
  \begin{itemize}
  \item input \lstinline{input = open('data', 'r')}
  \item read all \lstinline{S = input.read()}
  \item read N bytes \lstinline{S = input.read(N)}
  \item read next \lstinline{S = input.readline()}
  \item read in lists \lstinline{L = input.readlines()}
  \item output \lstinline{output = open('/tmp/spam', 'w')}
  \item write \lstinline{output.write(S)}
  \item write \lstinline{strings output.writelines(L)}
  \item close \lstinline{output.close()}
  \end{itemize}
\end{frame}


% %%%%%%%%%%%%%%%%%%%%%%%%%%%%%%%%%%%%%%%%%%%%%%%%%%%%%%%%%%%%%%%%%%%%%%%%%%%%%%%%%%%
% \begin{frame}[fragile]\frametitle{Unsupported Types}
  % \begin{itemize}
  % \item no Boolean type, use integers. But, `True', `False'
  % \item no char or single byte, use strings of length one or integers
  % \item no pointer
  % \item int vs. short vs. long, only one integer type (C long)
  % \item float vs. double, only one floating point type (C double)
  % \end{itemize}
% \end{frame}

%%%%%%%%%%%%%%%%%%%%%%%%%%%%%%%%%%%%%%%%%%%%%%%%%%%%%%%%%%%%%%%%%%%%%%%%%%%%%%%%%%%
\begin{frame}[fragile]\frametitle{Comparisons vs. Equality}
  \begin{itemize}
  \item \lstinline{L1 = [1, ('a', 3)]}
  \item \lstinline{L2 = [1, ('a', 3)]}
  \item \lstinline{L1 == L2} is 1
    \item The \lstinline{==} operator tests value equivalence
    \item \lstinline{L1 is L2} is 0

  \item The \lstinline{is} operator tests object identity
  \end{itemize}
\end{frame}

%%%%%%%%%%%%%%%%%%%%%%%%%%%%%%%%%%%%%%%%%%%%%%%%%%%%%%%%%%%%%%%%%%%%%%%%%%%%%%%%%%%
\begin{frame}[fragile]\frametitle{if, elif, else}
  \begin{lstlisting}
	if not done and (x > 1):
		doit()
	elif done and (x <= 1):
		dothis()
	else:
		dothat()
  \end{lstlisting}
\end{frame}

%%%%%%%%%%%%%%%%%%%%%%%%%%%%%%%%%%%%%%%%%%%%%%%%%%%%%%%%%%%%%%%%%%%%%%%%%%%%%%%%%%%
\begin{frame}[fragile]\frametitle{while, break}
  \begin{lstlisting}
	while 1:
		line = ReadLine()
		if len(line) == 0:
			break
  \end{lstlisting}
\end{frame}

%%%%%%%%%%%%%%%%%%%%%%%%%%%%%%%%%%%%%%%%%%%%%%%%%%%%%%%%%%%%%%%%%%%%%%%%%%%%%%%%%%%
\begin{frame}[fragile]\frametitle{for}
String:
  \begin{lstlisting}
	for letter in `hello world':
		print(letter)
  \end{lstlisting}
  List:
  \begin{lstlisting}
	for item in [12, 'test', 0.1+1.2J]:
		print(item)
  \end{lstlisting}
 Range with bounds and step:		
  \begin{lstlisting}
	for i in range(2,10,2):
		print(i)
  \end{lstlisting}
  Equivalent to the C loop:
    \begin{lstlisting}
	for (i = 2; i < 10; i+=2){
		printf("%d\n",i);
	}
  \end{lstlisting}
\end{frame}

%%%%%%%%%%%%%%%%%%%%%%%%%%%%%%%%%%%%%%%%%%%%%%%%%%%%%%%%%%%%%%%%%%%%%%%%%%%%%%%%%%%
\begin{frame}[fragile]\frametitle{pass}
Temporary filler, the stub.
  \begin{lstlisting}
pass
  \end{lstlisting}
  Functions, for loop, wherever there is ``:'', then on the indented next line $pass$ can be put.
\end{frame}

%%%%%%%%%%%%%%%%%%%%%%%%%%%%%%%%%%%%%%%%%%%%%%%%%%%%%%%%%%%%%%%%%%%%%%%%%%%%%%%%%%%
\begin{frame}[fragile]\frametitle{errors and exceptions}
  \begin{lstlisting}
try:
	f = open('blah')
except IOError:
	print(`could not open file')
  \end{lstlisting}
  \begin{itemize}
  \item NameError attempt to access an undeclared variable
  \item ZeroDivisionError division by any numeric zero
  \item SyntaxError Python interpreter syntax error
  \item IndexError request for an out-of-range index for sequence
  \item KeyError request for a non-existent dictionary key
  \item IOError input/output error
  \item AttributeError attempt to access an unknown object attribute
  \end{itemize}
\end{frame}


%%%%%%%%%%%%%%%%%%%%%%%%%%%%%%%%%%%%%%%%%%%%%%%%%%%%%%%%%%%%%%%%%%%%%%%%%%%%%%%%%%%
\begin{frame}[fragile]\frametitle{Functions}
  \begin{lstlisting}
def test(a,b=2,d=func):
    return d(a,b)
	
test(3)
test(b=4,a=3)
test(1,2,lambda x,y: x*y)
test(1,2,g)
  \end{lstlisting}
  \begin{itemize}
  \item Functions can return any type of object.
  \item When nothing is return the None object is returned by default.
  \item Multiple values can be returned.
  \item Anonymous functions ``lambda''.
  \item Parameters can have default arguments.
  \item Variable-length arguments are supported.
  \end{itemize}
\end{frame}

%%%%%%%%%%%%%%%%%%%%%%%%%%%%%%%%%%%%%%%%%%%%%%%%%%%%%%%%%%%%%%%%%%%%%%%%%%%%%%%%%%%
\begin{frame}[fragile]\frametitle{Modules, namespaces and packages}

  \begin{itemize}
  \item A file is a module, e.g. `myio.py', with a function 'load'
  \item To use that function from another file:
    \begin{lstlisting}
import myio
myio.load()
  \end{lstlisting}
  \item Code in 'myio.py' will be in the 'myio' namespace.
  \item Selective import:
      \begin{lstlisting}
from myio import load
load()
  \end{lstlisting}
\item  Packages are bundle of modules.
  \end{itemize}
\end{frame}

%%%%%%%%%%%%%%%%%%%%%%%%%%%%%%%%%%%%%%%%%%%%%%%%%%%%%%%%%%%%%%%%%%%%%%%%%%%%%%%%%%%
\begin{frame}[fragile]\frametitle{Class}
  \begin{lstlisting}
class Cone(SomeParantClass):
	def __init__(self,d0,de,L):
		self.a0 = d0/2
		self.ae = de/2
		self.L = L
	def __del__(self):
		pass
	def radius(self,z):
		return self.ae + (self.a0-self.ae)*z/self.L
	def radiusp(self,z):
		return (self.a0-self.ae)/self.L

c = Cone(0.1,0.2,1.5)
c.radius(0.5)
  \end{lstlisting}

\end{frame}

%%%%%%%%%%%%%%%%%%%%%%%%%%%%%%%%%%%%%%%%%%%%%%%%%%%%%%%%%%%%%%%%%%%%%%%%%%%%%%%%%%%
\begin{frame}[fragile]\frametitle{Standard library core modules}
  \begin{itemize}
  \item \textbf{os} file and process operations.
%  \item \textbf{os.path} platform-independent path and filename utilities
  \item \textbf{time} dates and times related functions.
  \item \textbf{string} commonly used string operations.
%  \item \textbf{math,cmath} math operations and constants, complex version
  \item \textbf{re} regular expressions.
%  \item \textbf{sys} access to interpreter variables
%  \item \textbf{gc} control over garbage collector
  \item \textbf{copy} allow to copy object.
  \end{itemize}
\end{frame}

%%%%%%%%%%%%%%%%%%%%%%%%%%%%%%%%%%%%%%%%%%%%%%%%%%%%%%%%%%%%%%%%%%%%%%%%%%%%%%%%%%%
\begin{frame}[fragile]\frametitle{other library modules}
  \begin{itemize}
  \item \textbf{Tkinter}: Tk GUI toolkit (cross-platform).
     \item \textbf{NumPy}: Numerical array processing.
     \item and many many more \ldots
     \item Visit https://pypi.python.org/pypi for a comprehensive listing.
   \end{itemize}
\end{frame}
