%%%%%%%%%%%%%%%%%%%%%%%%%%%%%%%%%%%%%%%%%%%%%%%%%%%%%%%%%%%
 \begin{frame}[fragile] \frametitle{Notes from Avinash Sathaye}
https://http://www.ms.uky.edu/~sohum/ma162/fa\_09/lectures/
\end{frame}

%%%%%%%%%%%%%%%%%%%%%%%%%%%%%%%%%%%%%%%%%%%%%%%%%%%%%%%%%%%%%%%%%%%%%%%%%%%%%%%%%%
 \begin{frame}[fragile]\frametitle{}
\begin{center}
{\Large Understanding monetary transactions.}

\end{center}
\end{frame}


\begin{frame} %2

  \frametitle{The Simple Interest.}
 \begin{itemize}%[<+-| alert@+>]  
\item 
Interest is \mbl{rent on borrowed money.} 

We use the \mbl{notation $P$ for principal}.
This is the amount of money borrowed (or lent).

\item
We say \mbl{ $r$ is the interest rate},
if it is the agreed interest(rent)  on one dollar
per year. Of course, we change the name of the currency as appropriate!


The rate is often quoted as something like $7$\% which means
$r=\frac{7}{100} = 0.07$. Thus the words \mbl{``per cent''}
mathematically mean the fraction $\frac{1}{100}$.

\item We shall typically let $t$ denote the period of lending in years
and thus  the interest $I$ accumulated in $t$ years on a principal of
$P$ dollars at a rate $r$ is given by the formula:
$$I = Prt.$$




\end{itemize}
%\pause 
\end{frame}

%2
%


\begin{frame}%3
  \frametitle{Simple Interest continued.}
  \begin{itemize}%[<+-| alert@+>]

\item 
Thus after the $t$ years, the total amount owed is the original
principal $P$ plus the interest $I$ and thus has the formula;

$$\mbox{ Accumulation } A = P + Prt = P(1+rt).$$

\item If we know three of the four quantities $A,P,r,t$ then we can find
the fourth. We should learn to recognize what is given and what is
unknown.

\item {\bf Example 1.} If you invest \$$770.91$ at $8$\% simple interest,
how much will your investment be worth in $15$ months?
\item
We note that $r=8$\% or $r=0.08$. Also $P=770.91$. We are given $t=15$
months which must be converted to years and thus $t=\frac{15}{12} =
1.25$. We are looking for $A$, the net accumulated value of the investment.
Hence
$$A=770.91(1+0.08\cdot 1.25) = 848.001. \mbox{ or \$$848$.}$$




\end{itemize}
%\pause
\end{frame}
%3
%


\begin{frame}%4
  \frametitle{More examples.}
  \begin{itemize}%[<+-| alert@+>]

\item
{\bf Example 2.}
If you invest \$$1127.52$ and after $18$ months it is worth \$$1229.00$,
what simple interest rate, expressed as a percentage and rounded to .01,
did you receive?
\item We are given $P=1127.52$, $t=\frac{18}{12}=1.5$ and $A=1229$.
We want $r$.

We recommend solving the formula $A=P(1+rt)$ for $r$ and then evaluating
it. Thus:
$$r=\frac{\frac{A}{P}-1}{t} = \frac{\frac{1229}{1127.52}-1}{1.5}.$$
The answer comes out $r=0.06000189206$. We multiply by $100$ to make a
per cent rate and report $r=6$\% after rounding.

\end{itemize}
%\pause
\end{frame}

%4


\begin{frame}%5
  \frametitle{Examples Continued.}
  \begin{itemize}%[<+-| alert@+>]

\item {\bf Example 3.}
If you invest \$$1520.88$ at $6$\% simple interest,
after how many months, rounded to $0.01$, will your investment be
worth \$$1657.00$?
\item We are given $P=1520.88$, $r=0.06$, $A=1657$ and asked to find
$t$.

As before, we solve our formula for $t$ and evaluate:
$$t= \frac{\frac{A}{P}-1}{r} = \frac{\frac{1657}{1520.88}-1}{0.06}.$$
This gives $t=1.491680255$. Be sure to multiply by $12$ to make months.
So the answer is $17.90016306$ or $17.9$ months after rounding. 



\end{itemize}
%\pause
\end{frame}

%5
%



\begin{frame}%6
  \frametitle{A More Complicated Example.}
  \begin{itemize}%[<+-| alert@+>]
 
\item {\bf Example 4.}
Homer won a prize in the lottery of \$$3000$, payable \$$1500$
immediately and \$$1500$ plus $4$\% simple interest payable in $260$
days.
Getting impatient, Homer sells the promissory note to Moe for \$$1440$
cash after $170$ days. Using a nominal $360$ day year, find the
simple interest rate, rounded to $0.01$, earned by Moe.
\item
This is a simple interest problem of finding $r$, but needs careful set
up. If Homer were to patiently wait the $260$ days, he would earn
$A=P(1+rt) = 1500(1+0.04\cdot \frac{260}{360}) = 1543.33 \mbox{
dollars }.$

\item \mbl{From Moe's perspective,} this is his $A$ after a lending of \$$1440$
for a period of $260-170 = 90$ days. Thus for Moe, the calculated
interest rate as in Example 2 is 
$r=\frac{\frac{A}{P}-1}{t} = \frac{\frac{1543.33}{1440}-1}{\frac{90}{360}}
= 0.2870370361 = 28.7\%.$
 Such high rates
are not uncommon for short term lenders!


\end{itemize}
%\pause
\end{frame}

%6
%


\begin{frame}%7
  \frametitle{The Greedy Lender.}
  \begin{itemize}%[<+-| alert@+>]
 

\item Suppose you lend somebody \$$100$ for a period of one year at
$10$\% interest rate. You will receive an accumulated payback of \$$110$
at the end of the year.

But if you demand a repayment in six months, you will be entitled to
receive \$$105$. Now, suppose you lend this total amount back to the
borrower, then using the usual formula with $P=105, r=0.10, t=0.5$ we
get $105(1+0.10\cdot 0.5) = 110.25$ dollars!

It is easy to see that the net formula is $100(1+\frac{0.10}{2})^2$.

\item Of course, you don't really want to carry out the transaction,
just demand the money. This is called the accumulation by
\mbl{compounding every six months or twice a year!}

Thus a greedy lender can claim more money by simply ``imagining'' a
transaction!



\end{itemize}
%\pause
\end{frame}



%7
%

\begin{frame}%8
 \frametitle{The Compound Interest.}
  \begin{itemize}%[<+-| alert@+>]
 
\item 
If one gets greedier and imagines compounding the interest $m$ times a
year then it is easy to see that in each of the $m$ periods, we get the
accumulation by multiplying the starting principal for that period by
$(1+\frac{r}{m})$ and thus the full formula for the interest after one
year is: 
$$A=P\left(1+\frac{r}{m}\right)^m.$$

\item It is useful to develop some {\bf new notation.}

Assume that we are \mbl{compounding $m$ times each year.} Thus in $t$
years, we shall have $mt$ periods of compounding and we define:

\mbl{ Periodic interest rate $i=\frac{r}{m}$} and \mbl{ Total term of
loan in periods $n=tm$.}

\item This gives us the formula for accumulated amount when we compound
$m$ times a year as:
$$A = P\left(1+\frac{r}{m}\right)^{mt} = P(1+i)^n.$$



\end{itemize}
%\pause
\end{frame}

%8
%



\begin{frame}%9
 \frametitle{This Greed has a Limit!}
  \begin{itemize}%[<+-| alert@+>]
 
\item Continuing our example of lending \$$100$ for one year at a rate
of $10$\%. If we compound it $m$ times a year, then we
have the formula $A_m = 100\left(1+\frac{0.10}{m}\right)^m$.

\item We can calculate the
accumulation for different values of $m$:
$$\begin{array}{|l|l|l|l|l|}
\hline
m & 1 & 11 &   51 &  101\\\hline
A_m & 110.00 & 110.46717 & 110.50627 & 110.51162\\\hline
\end{array}
$$

\item Thus, though increasing, it is not growing very fast. Indeed,
using techniques of algebra it is possible to show that the limit of the
quantity $A_m$ as $m$ goes to infinity is a famous function of
mathematics, namely
$$\lim_{m\rightarrow \infty} P\left(1+\frac{r}{m}\right)^m = P\exp(r).$$
Thus, even if we imagine infinite compounding, our accumulation for the
above $P=100, r=10\%$ is only $100\exp(0.10) = 110.5170918$.

  
\end{itemize}
%\pause
\end{frame}

%9
%


\begin{frame}%10
  \frametitle{The Compound Interest Formulas.}
  \begin{itemize}%[<+-| alert@+>]

\item To summarize, we have the formula that for principal $P$, annual rate
$r$, period $t$ years and compounded $m$ times a year, we have
$$A_m =P(1+i)^n \mbox{ where } i=\frac{r}{m}, ~ n= tm .$$

\item We describe the idea of infinite compounding as
\mbl{continuous compounding.}
The accumulation if we \mbl{compound continuously} is given by the formula:
$$A_C = P\exp(rt).$$
\item Typically, we just write $A$ for accumulation, but mention the
value of $m$ in words.


\end{itemize}
%\pause
\end{frame}
%10
%



\begin{frame}%11
  \frametitle{Examples of Compound Interest.}
  \begin{itemize}%[<+-| alert@+>]
\item {\bf Example 5.}
If you invest \$$5000.00$ at $9$\% compounded bi-weekly, how much
will your investment be worth in $8$ years?
\item We have $P=5000, r=0.09, t=8$. The meaning of the phrase bi-weekly
is that it is compounded once every two weeks or $m=26$ using a nominal
year of $52$ weeks.

We have $i=\frac{0.09}{26} = 0.003461538462$ and $n=8\cdot 26 = 208$.

Thus $$A=5000(1+0.003461538462)^{208} = 10259.40275.$$
\item 
\mrd{Warning:} It is crucial to learn good calculator techniques here,
since if you don't keep enough accuracy for  $i$, then the power
calculation introduces a lot of error and multiplication by a large $P$
makes a very inaccurate amount. One should try not to copy down
intermediate results, but store and reuse them for better accuracy!


\end{itemize}
%\pause
\end{frame}
%

%11

\begin{frame}%12
  \frametitle{ More Examples.}
  \begin{itemize}%[<+-| alert@+>]

\item {\bf Example 6.}
How much did you invest at $8$\% compounded bi-weekly if $15$ years
later the investment is worth \$$97000.00$?

\item If the investment is $P$ then our formula gives:
$$97000 = P\left(1+\frac{0.08}{26}\right)^{(15\cdot 26)} $$
which can be solved for $P$ as:
$$P = 97000\cdot \left(\left(1+\frac{0.08}{26}\right)^{(-15\cdot 26)}\right).$$
\item
This evaluates to $29269.71472$. It is an excellent idea to double check that
this value of $P$ does generate the $97000$ i.e.
$$29269.71472\left(1+\frac{0.08}{26}\right)^{15\cdot 26} = 97000$$
within reasonable accuracy! The computer answer is $96999.99712$.

\end{itemize}
%\pause
\end{frame}
%12


\begin{frame}%13
  \frametitle{Effective Rate.}
  \begin{itemize}%[<+-| alert@+>]
\item  Often, lending terms are described by different rates and
different number of compoundings per year. It is necessary to be able to
compare them to decide which is a better rate.

\item One way to do this is to find an \mbl{effective rate $r_{eff}$},
which is defined as a simple interest rate which will give the same
yield as the given scheme.

\item Thus, if we invest one dollar at $r$\% annual rate compounded $m$
times a year, then our net yield is $(1+\frac{r}{m})^m$ and if $r_{eff}$
is to be the effective rate, then we have:
$$\left(1+\frac{r}{m}\right)^m = 1+r_{eff} $$
so we have the formula;
$$r_{eff} = \left(1+\frac{r}{m}\right)^m -1.$$


\end{itemize}
%\pause
\end{frame}

%13
%
\begin{frame}%14
  \frametitle{Example of Effective Rate.}
  \begin{itemize}%[<+-| alert@+>]
\item {\bf Example 7.}
Bank A is offering an interest rate of $6.60$\% compounded monthly,
while bank B is offering an interest rate of $6.69$\% compounded quarterly.

What are the effective rates of the two banks expressed as percents and 
for the investor, which bank offers the bettter rate?

\item We apply the formula for the effective rate to get:

The $r_{eff}$ for bank A is:
$\left(1+\frac{0.066}{12}\right)^{12} -1 = 0.06803356$\\
and the $r_{eff}$ for bank B is:
$\left(1+\frac{0.0669}{4}\right)^{4} -1 =0.06859714600.$
\item The reported answers should be $6.80$\% and $6.86$\% respectvely,
with bank B declared as having a better rate.
\item Note that if the problem was about borrowing from the bank
instead of investing, then bank A would be a better choice!!

\end{itemize}
%\pause
\end{frame}
%14
%

\begin{frame}%15
  \frametitle{Preview.}
  \begin{itemize}%[<+-| alert@+>]
\item Next, we study the concepts of progression or a sequence and a
series (or their sum).

Of special interest are the Arithmetic series and the Geometric series;
\mbl{a must study } for all students of mathematics!

\item Afterwards, we tackle a problem of annuity. Such problems have three types.
\item We discuss how to  borrow a large sum and pay back with periodic payments
(mortgage).
\item
We discuss  how much money can be
withdrawn on a periodic basis from set up funds which are earning
interest until drawn (sinking funds).

\item  We also dicuss how to build up future reserves by periodic
saving.
 

\end{itemize}
\end{frame}

%%%%%%%%%%%%%%%%%%%%%%%%%%%%%%%%%%%%%%%%%%%%%%%%%%%%%%%%%%%%%%%%%%%%%%%%%%%%%%%%%%
 \begin{frame}[fragile]\frametitle{}
\begin{center}
{\Large Understanding Annuities}

\end{center}
\end{frame}

\begin{frame} %2

  \frametitle{Some Algebraic Terminology.}
 \begin{itemize}%[<+-| alert@+>]  
\item We recall some terms and calculations from elementary algebra.

\item A finite sequence of numbers is a function of natural numbers $1,2,\cdots , n$.
Thus, the formula $a_k=2k+1$ for $k=1,2,\cdots , 10$ describes a
sequence $3,5,7,9,11,13,15,17,19,21$.

\item We may also let a sequence run out to infinity as in
$1, \frac{1}{2}, \frac{1}{3},\cdots ,\frac{1}{n}, \cdots $.
Here the sequence can also be described as $\frac{1}{n}$ where
$n=1,2,\cdots $.

\item A sequence may also be called a \mbl{progression}.
Two progressions are important, the Arithmetic Progression and the
Geometric Progression.




\end{itemize}
%\pause 
\end{frame}

%2
%

\begin{frame}%3
  \frametitle{A.P. and G.P.}
  \begin{itemize}%[<+-| alert@+>]

\item {Arithmetic progression:} This is a sequence which has
\mbl{a starting
number $a$} and successive numbers are obtained by adding a number
\mbl{ $d$ (called the common difference.)}

Thus, its $n$-th term is $a+(n-1)d$.

{\bf Example:} Take $a=3, d=4$. The sequence is
$$3,7,11,15,19,\cdots , 3+4(n-1),\cdots .$$
The $n$-th term can  be better written as $4n-1$.

\item {Geometric progression:} The geometric progression has
\mbl{a starting
number $a$} and successive terms are obtained by multiplying by
\mbl{a common
ratio $r$.}

Thus, its $n$-th term is $ar^{(n-1)}$.

{\bf Example:} Take $a=2$ and $r=\frac{1}{2}$.
The sequence is:
$$2, 1, \frac{1}{2}, \frac{1}{4},\cdots , \frac{2}{2^{(n-1)}}, \cdots .$$
Note that the $n$-th term is better written as $\frac{1}{2^{(n-2)}}$.


\end{itemize}
%\pause
\end{frame}
%3
%


\begin{frame}%4
  \frametitle{Arithmetic Series.}
  \begin{itemize}%[<+-| alert@+>]

\item We need the formula  for the sum of terms in A.P. 

\item The sum of the A.P. $$a,a+d, a+2d,\cdots , a+(n-1)d$$
is called an Arithmetic Series and is written as
$\displaystyle \sum_{k=1}^n (a+(k-1)d)$.

\item
Its sum is given by the formula:
$$\sum_{k=1}^n a+(k-1)d = n\frac{a+a+(n-1)d}{2}=n\left(a+\frac{n-1}{2}d\right).$$
An alternate way to remember it is
$\mbox{( number of terms )}\cdot \mbox{( average of the first and the last term )}.$

\end{itemize}
%\pause
\end{frame}

%4
%


\begin{frame}%5
  \frametitle{Geometric Series.}
  \begin{itemize}%[<+-| alert@+>]
 
\item We need the formulas  for the sum of terms in G.P. 

\item The sum of the G.P. $$a,ar, ar^2,\cdots , ar^{(n-1)}$$
is called a  Geometric Series and is written as
$\displaystyle \sum_{k=1}^n (ar^{(k-1)})$.

\item
Its sum is given by the formula:
$$\sum_{k=1}^n (ar^{(k-1)}) =a\left(\frac{r^n-1}{r-1}\right)
= a\left(\frac{1-r^n}{1-r}\right).$$
\item If $|r|<1$, then we can make sense of the formula even for an
infinite G.P. and write;
$$\sum_{k=1}^\infty (ar^{(k-1)}) =a\left(\frac{1}{1-r}\right).$$

\end{itemize}
%\pause
\end{frame}

%5
%


\begin{frame}%6
  \frametitle{Basic Annuity.}
  \begin{itemize}%[<+-| alert@+>]
 
\item What is an annuity?  An annuity is  a combination of investments
(or payments).
\item For convenience,\mbl{ we assume the following conditions} which
are valid in most practical situations.
\item A fixed amount $R$ is invested exactly $m$ times a year.
This gives exactly $m$ periods in a year and each is $\frac{1}{m}$-th
part of the year.
\item Each payment is made at the end of its period.
\item The payments are made for a period of $t$ years and thus the
\mbl{ number of payments } is exactly $mt=n$.
\item For each period, the interest rate is the same $r$\% annual and
thus in each period, the interest earned by $1$ dollar is exactly
$\frac{r}{m} = i$. This is called \mbl{ the periodic rate.}

\end{itemize}
%\pause
\end{frame}

%5
%

\begin{frame}%7
  \frametitle{Basic Annuity Formula.}
  \begin{itemize}%[<+-| alert@+>]

\item With the notation as explained above, how much money will be
accumulated by making a periodic investment of $R$ dollars at the end of
each of the  $n$ periods when the periodic rate is $i$ and the interest
is compounded in each period?

\item The answer comes out as a geometric series.
Here is how we reason it out.
\item The payment at the end of the first period is compounded for
$(n-1)$ periods and hence becomes worth $R(1+i)^{(n-1)}$.

\item The payment at the end of the second period is compounded only for
$(n-2)$ periods and becomes worth $R(1+i)^{(n-2)}$.
\item Continuing, the very last payment is worth $R(1+i)^{(n-n)} = R$.
In other words, it acquires no interest!
\item Adding up the terms in reverse,$S=R + R(1+i) + \cdots +
R(1+i)^{(n-1)}$, or $S= R \frac{(1+i)^n -1}{(1+i)-1}
= R \frac{(1+i)^n -1}{i}.$


\end{itemize}
%\pause
\end{frame}

%6
%

\begin{frame}%7
  \frametitle{Present Value of an Annuity.}
  \begin{itemize}%[<+-| alert@+>]
 

\item
Often, the periodic investments are just payments - like mortgage -
against borrowed funds. What is the relation between the periodic
payment $R$ and the borrowed amount $P$, when the interest rate is $r$\%
and the payment is $m$ times a year?

\item

As usual, we let $i$ be the periodic rate and $n$ the number of periods
or the total number of payments.

Think like the lender and find out what single investment  of $P$ dollars
would yield the same accumulation in same number of years and same rate.
\item This gives us the equation:
$\mbl{P(1+i)^n =S =  R \frac{(1+i)^n -1}{i}}$
and thus the formula:
$\mbl{P} = R \frac{(1+i)^n -1}{i (1+i)^n} = \mbl{R\frac{1-(1+i)^{(-n)}}{i}}.$

This gives the needed formula
$\mbl{R = P\frac{i}{1-(1+i)^{(-n)}}} .$


\end{itemize}
%\pause
\end{frame}

%7
%



\begin{frame}%8
 \frametitle{Using the Annuity Formulas.}
  \begin{itemize}%[<+-| alert@+>]
 
\item 
We now have the basic formulas needed to answer all questions about
periodic investments or payments.

\item \mbl{Example of a Trust Fund} If a trust is set up
so that you take  $6$ years to travel and pursue other interests.

Suppose that you will make  bi-weekly withdrawals of \$$2,000$ from a
money market account that pays $4.00$\% compounded bi-weekly.

How much should the fund be?

\item {\bf Answer:} Imagine the trust fund to be a lender and your
withdrawls as mortgage payments to you. Thus, we use the formula:

$\mbl{P = R\frac{1-(1+i)^{(-n)}}{i}.}$
Here $R=2000$, $i=\frac{4}{2600}$ and $n=26\cdot 6 = 156$.

The formula yields $277195.1659$ or \$$277,195.17$.
 

\end{itemize}
%%\pause
\end{frame}

%8
%



\begin{frame}%9
 \frametitle{More Examples of Annuities.}
  \begin{itemize}%[<+-| alert@+>]
 
\item
\mbl{Sinking Fund.} This means a fund set up with periodic investments
to be sunk or used up at the end of the $n$ periods.

\item \mbl{Example.}
You plan on buying equipment worth $30,000$ dollars in $3$ years.
 Since you firmly believe in not borrowing, you plan on making
 monthly payments into an account that pays $4.00$\% compounded monthly .
 How much must your payment be?

\item You have to find out the value of $R$, but know that $S$, the
expected accumulation is $30,000$ with $t=3$ and $r=0.04$.

\item Moreover $m=12$ (from the word monthly!!) and hence
$i=\frac{0.04}{12}=.003333$ and $n=12\cdot 3 = 36$.

\item  Using
$S=R \frac{(1+i)^n -1}{i}$ 
 we get
$\mbl{R = 30000\left( \frac{i}{(1+i)^n -1}\right) = 785.7195502.}$

\item Thus, the reported answer is $785.72$ which actually yields
\$$30000.02$.

\end{itemize}
%%\pause
\end{frame}

%9
%


\begin{frame}%10
  \frametitle{Continued Examples.}
  \begin{itemize}%[<+-| alert@+>]

\item \mbl{About Accuracy.}
In the above calculation, the evaluation of
$$\frac{(1+i)^n -1}{i} = \frac{(1+0.003333)^{36} -1}{0.003333}$$
is involved. If you calculate this and divide into $30000$, you need to
keep many digits of accuracy. Try various approximations to see how to
get the most accurate answer (to the penny).

\item You will find that you need to keep at least four accurate decimal
places the the first answer.

\item Thus, as a general principle, in these problems, you should not
copy down intermediate answers, but store and recall them, so that
maximum accuracy is maintained.

\end{itemize}
%%\pause
\end{frame}
%10
%


\begin{frame}%11
  \frametitle{Further Examples of Annuity.}
  \begin{itemize}%[<+-| alert@+>]
\item
As another example, consider this problem.

\item If you can afford a monthly payment of \$$1010$ for $33$ years and
if the available
interest rate is $4.10$\%, what is the maximum amount
that you can afford to borrow?

\item You note that $R=1010$, $i=\frac{r}{m}=\frac{0.041}{12}$ and
$m=12$ with $t=33$, so that $n=12\cdot 33 = 396$.

\item But you don't want $S$, the future accumulation! You want the
money now, to be paid back over the years. So, you use the formula for
$P$, the present value.

\item Thus, you evaluate:
$\mbl{P = R\frac{1-(1+i)^{(-n)}}{i} = 1010\cdot 216.8603683 = 219028.97.}$

Note that due to the large numbers involved, your fraction needs 10
digit accuracy!

\item Thus, the hardest part is always to figure out which formula is
appropriate!
\end{itemize}
%%\pause
\end{frame}
%

%11

\begin{frame}%12
  \frametitle{ More Examples.}
  \begin{itemize}%[<+-| alert@+>]

\item {\bf Example 6.}
How much did you invest at $8$\% compounded bi-weekly if $15$ years
later the investment is worth \$$97000.00$?

\item If the investment is $P$ then our formula gives:
$$97000 = P\left(1+\frac{0.08}{26}\right)^{(15\cdot 26)} $$
which can be solved for $P$ as:
$\mbl{P = 97000\cdot \left(\left(1+\frac{0.08}{26}\right)^{(-15\cdot 26)}\right).}$
\item
This evaluates to $29269.71472$. It is an excellent idea to double check that
this value of $P$ does generate the $97000$ i.e.
$\mbl{29269.71472\left(1+\frac{0.08}{26}\right)^{15\cdot 26} = 97000}$
within reasonable accuracy! The computer answer is $96999.99712$.

\end{itemize}
%%\pause
\end{frame}
%12


\begin{frame}%13
  \frametitle{Effective Rate.}
  \begin{itemize}%[<+-| alert@+>]
\item  Often, lending terms are described by different rates and
different number of compoundings per year. It is necessary to be able to
compare them to decide which is a better rate.

\item One way to do this is to find an \mbl{effective rate $r_{eff}$},
which is defined as a simple interest rate which will give the same
yield as the given scheme.

\item Thus, if we invest one dollar at $r$\% annual rate compounded $m$
times a year, then our net yield is $(1+\frac{r}{m})^m$ and if $r_{eff}$
is to be the effective rate, then we have:
$$\left(1+\frac{r}{m}\right)^m = 1+r_{eff} $$
so we have the formula;
$\mbl{r_{eff} = \left(1+\frac{r}{m}\right)^m -1.}$


\end{itemize}
%%\pause
\end{frame}

%13
%
\begin{frame}%14
  \frametitle{Example of Effective Rate.}
  \begin{itemize}%[<+-| alert@+>]
\item {\bf Example 7.}
Bank A is offering an interest rate of $6.60$\% compounded monthly,
while bank B is offering an interest rate of $6.69$\% compounded quarterly.

What are the effective rates of the two banks expressed as percents and 
for the investor, which bank offers the bettter rate?

\item We apply the formula for the effective rate to get:

The $r_{eff}$ for bank A is:
$\left(1+\frac{0.066}{12}\right)^{12} -1 = 0.06803356$\\
and the $r_{eff}$ for bank B is:
$\left(1+\frac{0.0669}{4}\right)^{4} -1 =0.06859714600.$
\item The reported answers should be $6.80$\% and $6.86$\% respectvely,
with bank B declared as having a better rate.
\item Note that if the problem was about borrowing from the bank
instead of investing, then bank A would be a better choice!!

\end{itemize}
%%\pause
\end{frame}
%14
%

\begin{frame}%15
  \frametitle{Preview.}
  \begin{itemize}%%[<+-| alert@+>]
\item Next, we study the concepts of progression or a sequence and a
series (or their sum).

Of special interest are the Arithmetic series and the Geometric series;
\mbl{a must study } for all students of mathematics!

\item Afterwards, we tackle a problem of annuity. Such problems have three types.
\item We discuss how to  borrow a large sum and pay back with periodic payments
(mortgage).
\item
We discuss  how much money can be
withdrawn on a periodic basis from set up funds which are earning
interest until drawn (sinking funds).

\item  We also dicuss how to build up future reserves by periodic
saving.
 

\end{itemize}
\end{frame}

