%%%%%%%%%%%%%%%%%%%%%%%%%%%%%%%%%%%%%%%%%%%%%%%%%%%%%%%%%%%%%%%%%%%%%%%%%%%%%%%%%%
\begin{frame}[fragile]\frametitle{}

\begin{center}
{\Large NER in NLTK}
\end{center}
\end{frame}

%%%%%%%%%%%%%%%%%%%%%%%%%%%%%%%%%%%%%%%%%%%%%%%%%%%%%%%%%%%%%%%%%%%%%%%%%%%%%%%%%%
\begin{frame}[fragile]\frametitle{Steps for NER}
  \begin{itemize}
  \item Take a string as input
  \item Tokenize it into sentences
  \item Tokenize the sentences into words
  \item Add part-of-speech tags to the words using \lstinline|nltk.pos_tag()|
  \item  Run this through the NLTK-provided NER classifier using \lstinline|nltk.ne_chunk()|
  \item  Parse these intermediate results and return any extracted entities

  \end{itemize}
\end{frame}


%%%%%%%%%%%%%%%%%%%%%%%%%%%%%%%%%%%%%%%%%%%%%%%%%%%%%%%%%%%%%%%%%%%%%%%%%%%%%%%%%%
\begin{frame}[fragile]\frametitle{NLTK NER Chunker}
  \begin{itemize}
  \item \lstinline|ne_chunk| needs part-of-speech annotations to add NE labels to the sentence. The output of the \lstinline|ne_chunk| is a \lstinline|nltk.Tree| object.
  \begin{lstlisting}
from nltk import word_tokenize, pos_tag, ne_chunk
sentence = "Mark and John are working at Google."
print ne_chunk(pos_tag(word_tokenize(sentence)))
"""
(S
  (PERSON Mark/NNP)
  and/CC
  (PERSON John/NNP)
  are/VBP
  working/VBG
  at/IN
  (ORGANIZATION Google/NNP)
  ./.)
"""
  \end{lstlisting}
  \end{itemize}
\end{frame}


%%%%%%%%%%%%%%%%%%%%%%%%%%%%%%%%%%%%%%%%%%%%%%%%%%%%%%%%%%%%%%%%%%%%%%%%%%%%%%%%%%
\begin{frame}[fragile]\frametitle{Steps for NER}
  \begin{itemize}
  \item NLTK provides a classifier that has already been trained to recognize named entities, accessed with the function \lstinline|nltk.ne_chunk()|. 
\item If we set the parameter \lstinline|binary=True|, then named entities are just tagged as \lstinline|NE|; otherwise, the classifier adds category labels such as \lstinline|PERSON, ORGANIZATION, and GPE|.
  \begin{lstlisting}
 >>> print(nltk.ne_chunk(sent)) 
(S
  The/DT
  (GPE U.S./NNP)
  is/VBZ
  one/CD
  ...
  according/VBG
  to/TO
  (PERSON Brooke/NNP T./NNP Mossman/NNP)
  ...)
  \end{lstlisting}
  \end{itemize}
\end{frame}

%%%%%%%%%%%%%%%%%%%%%%%%%%%%%%%%%%%%%%%%%%%%%%%%%%%%%%%%%%%%%%%%%%%%%%%%%%%%%%%%%%
\begin{frame}[fragile]\frametitle{IOB tagging}
The IOB Tagging system contains tags of the form:
  \begin{itemize}
  \item \lstinline|B-{CHUNK_TYPE}| - for the word in the Beginning chunk
  \item \lstinline|I-{CHUNK_TYPE}| - for words Inside the chunk
  \item \lstinline|O| - Outside any chunk
  \end{itemize}

\end{frame}

%%%%%%%%%%%%%%%%%%%%%%%%%%%%%%%%%%%%%%%%%%%%%%%%%%%%%%%%%%%%%%%%%%%%%%%%%%%%%%%%%%
\begin{frame}[fragile]\frametitle{IOB tagging}
  \begin{lstlisting}
from nltk.chunk import conlltags2tree, tree2conlltags
sentence = "Mark and John are working at Google."
ne_tree = ne_chunk(pos_tag(word_tokenize(sentence)))
iob_tagged = tree2conlltags(ne_tree)
print iob_tagged
"""[('Mark', 'NNP', u'B-PERSON'), ('and', 'CC', u'O'), ('John', 'NNP', u'B-PERSON'), ('are', 'VBP', u'O'), ('working', 'VBG', u'O'), ('at', 'IN', u'O'), ('Google', 'NNP', u'B-ORGANIZATION'), ('.', '.', u'O')]
"""
ne_tree = conlltags2tree(iob_tagged)
print ne_tree
""" (S
  (PERSON Mark/NNP)
  and/CC
  (PERSON John/NNP)
  are/VBP
  working/VBG
  at/IN
  (ORGANIZATION Google/NNP)
  ./.)
"""
\end{lstlisting}
\end{frame}

%%%%%%%%%%%%%%%%%%%%%%%%%%%%%%%%%%%%%%%%%%%%%%%%%%%%%%%%%%%%%%%%%%%%%%%%%%%%%%%%%%%
%\begin{frame}[fragile]\frametitle{Max Entropy}
%  \begin{lstlisting}
%import nltk
%nltk.download('maxent_ne_chunker')
%nltk.download('words')
%import re
%import time
%contentArray =['Starbucks is not doing very well lately.',
%               'Overall, while it may seem there is already a Starbucks on every corner, Starbucks still has a lot of room to grow.',
%               'Increase in supply... well you know the rules...',]
%for item in contentArray:
%            tokenized = nltk.word_tokenize(item)
%            tagged = nltk.pos_tag(tokenized)
%            #print tagged
%             namedEnt = nltk.ne_chunk(tagged)
%            namedEnt.draw()
%  \end{lstlisting}
%\end{frame}

%%%%%%%%%%%%%%%%%%%%%%%%%%%%%%%%%%%%%%%%%%%%%%%%%%%%%%%%%%%%%%%%%%%%%%%%%%%%%%%%%%
\begin{frame}[fragile]\frametitle{Stanford NER}
  \begin{lstlisting}
wget "http://nlp.stanford.edu/software/stanford-ner-2014-06-16.zip"
unzip stanford-ner-2014-06-16.zip
mv stanford-ner-2014-06-16 stanford-ner
sudo mv stanford-ner /usr/share/

from nltk import word_tokenize
from nltk.tag.stanford import NERTagger
 
classifier = '/usr/share/stanford-ner/classifiers/english.all.3class.distsim.crf.ser.gz'
jar = '/usr/share/stanford-ner/stanford-ner.jar'
st = NERTagger(classifier,jar)
sentence = word_tokenize("Rami Eid is studying at Stony Brook University in NY")
print st.tag(sentence)
  \end{lstlisting}
\end{frame}

%%%%%%%%%%%%%%%%%%%%%%%%%%%%%%%%%%%%%%%%%%%%%%%%%%%%%%%%%%%%%%%%%%%%%%%%%%%%%%%%%%
\begin{frame}[fragile]\frametitle{Training your own system}
  \begin{itemize}
  \item The feature extraction works almost identical as the one implemented in the Training a Part-Of-Speech Tagger, except need to add many features.
  \item Since the previous IOB tag is a very good indicator of what the current IOB tag is going to be, we have included the previous IOB tag as a feature.
  \item Spacy Provides Gold Parse method
  \item CRF++ can be used to generate custom NER tags.
  \end{itemize}
\end{frame}



