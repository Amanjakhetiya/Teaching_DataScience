%%%%%%%%%%%%%%%%%%%%%%%%%%%%%%%%%%%%%%%%%%%%%%%%%%%%%%%%%%%%%%%%%%%%%%%%%%%%%%%%%%
\begin{frame}[fragile]\frametitle{}
\begin{center}
{\Large Input/Output}
\end{center}
\end{frame}


%%%%%%%%%%%%%%%%%%%%%%%%%%%%%%%%%%%%%%%%%%%%%%%%%%%%%%%%%%%%%%%%%%%%%%%%%%%%%%%%%%%
\begin{frame}[fragile]\frametitle{Asking the user for input}
\begin{itemize}
\item  Python provides a built-in function called input that gets input from
the keyboard:
\item When the user presses Return or Enter, the program
resumes and input returns what the user typed as a string.
\begin{lstlisting}
>>> input = input()
Some silly stuff
>>> print(input)
Some silly stuff
\end{lstlisting}
\item Before getting input from the user, it is a good idea to print a prompt telling the
user what to input:
\begin{lstlisting}
>>> speed = input(prompt)
What...is the airspeed velocity of an unladen swallow?
What do you mean, an African or a European swallow?
>>> int(speed)
ValueError: invalid literal for int() with base 10:
\end{lstlisting}
\end{itemize}
\end{frame}

%%%%%%%%%%%%%%%%%%%%%%%%%%%%%%%%%%%%%%%%%%%%%%%%%%%%%%%%%%%%%%%%%%%%%%%%%%%%%%%%%%%
\begin{frame}[fragile]\frametitle{Prompt}
\begin{itemize}
\item  Combine prompt and input
\begin{lstlisting}
>>> prompt = 'What...is the airspeed velocity of an unladen swallow?\n'
>>> speed = input(prompt)
What...is the airspeed velocity of an unladen swallow?
17
>>> int(speed)
17
>>> int(speed) + 5
22
\end{lstlisting}
\item But if the user types something other than a string of digits, you get an error:
\begin{lstlisting}
>>> name = input('What is your name?\n')
What is your name?
Chuck
>>> print(name)
Chuck
\end{lstlisting}
\end{itemize}
\end{frame}

%%%%%%%%%%%%%%%%%%%%%%%%%%%%%%%%%%%%%%%%%%%%%%%%%%%%%%%%%%%%%%%%%%%%%%%%%%%%%%%%%%%
\begin{frame}[fragile]\frametitle{File IO}
\begin{itemize}
\item Can read data from a file, or write into a file
\item The open function takes two arguments, the name of the file, and the mode. The modes are:
\begin{lstlisting}
    'r': open a file for reading
    'w': open a file for writing. Caution: this will overwrite any previously existing file
    'a': append. Write to the end of a file.
\end{lstlisting}
\item The function returns a file object that performs the various tasks you'll be performing: a\_file = open(filename, mode). 
\begin{lstlisting}
    file.read(): read the entire contents of a file into a string
    file.write(some_string): writes to the file, note this doesn't automatically include any new lines.
    file.flush(): write out any buffered writes
    file.close(): close the open file. 
\end{lstlisting}
\end{itemize}
\end{frame}

%%%%%%%%%%%%%%%%%%%%%%%%%%%%%%%%%%%%%%%%%%%%%%%%%%%%%%%%%%%%%%%%%%%%%%%%%%%%%%%%%%%
\begin{frame}[fragile]\frametitle{Writing a file to disk}
\begin{lstlisting}
# Create the file temp.txt, and get it ready for writing
f = open("temp.txt", "w")
f.write("This is my first file! The end!\n")
f.write("Oh wait, I wanted to say something else.")
f.close()

# Create a file numbers.txt and write the numbers from 0 to 24 there
f = open("numbers.txt", "w")
for num in range(25):
    f.write(str(num)+'\n')
f.close()
\end{lstlisting}
\end{frame}


%%%%%%%%%%%%%%%%%%%%%%%%%%%%%%%%%%%%%%%%%%%%%%%%%%%%%%%%%%%%%%%%%%%%%%%%%%%%%%%%%%%
\begin{frame}[fragile]\frametitle{Reading a file from disk}
\begin{lstlisting}
# We now open the file for reading
f2 = open("temp.txt", "r")
# And we read the full content of the file in memory, as a big string
f2_content = f2.read()
f2.close()

# Read the file in the cell above, the content is in f2_content

# Split the content of the file using the newline character \n
lines = f2_content.split("\n")

# Iterate through the line variable (it is a list of strings)
# and then print the length of each line
for line in lines:
    print("Length of line `{0}` is {1}".format(line,len(line)))
\end{lstlisting}
\end{frame}

%%%%%%%%%%%%%%%%%%%%%%%%%%%%%%%%%%%%%%%%%%%%%%%%%%%%%%%%%%%%%%%%%%%%%%%%%%%%%%%%%%%
\begin{frame}[fragile]\frametitle{Reading a file from disk}
\begin{lstlisting}
# We now open the file for reading
f2 = open("numbers.txt", "r")
# And we read the full content of the file in memory, as a big string
f2_content = f2.read()
f2.close()

lines = f2_content.split("\n")
print(lines)

numbers = [int(line) for line in lines if len(line)>0]
print(numbers)
\end{lstlisting}
\end{frame}

%%%%%%%%%%%%%%%%%%%%%%%%%%%%%%%%%%%%%%%%%%%%%%%%%%%%%%%%%%%%%%%%%%%%%%%%%%%%%%%%%%%
\begin{frame}[fragile]\frametitle{Using try, except, and open}
\begin{lstlisting}
fname = input('Enter the file name: ')
try:
	fhand = open(fname)
except:
	print('File cannot be opened:', fname)
	exit()
count = 0
for line in fhand:
	if line.startswith('Subject:'):
		count = count + 1
print('There were', count, 'subject lines in', fname)
\end{lstlisting}
\end{frame}

%%%%%%%%%%%%%%%%%%%%%%%%%%%%%%%%%%%%%%%%%%%%%%%%%%%%%%%%%%%%%%%%%%%%%%%%%%%%%%%%%%%
\begin{frame}[fragile]\frametitle{Searching through a file}
\begin{lstlisting}
fhand = open('mbox-short.txt')
count = 0
for line in fhand:
	line = line.rstrip()
	if line.startswith('From:'):
		print(line)
\end{lstlisting}
\end{frame}
%
%
%%%%%%%%%%%%%%%%%%%%%%%%%%%%%%%%%%%%%%%%%%%%%%%%%%%%%%%%%%%%%%%%%%%%%%%%%%%%%%%%%%%%
%\begin{frame}[fragile]\frametitle{File I/O, I}
%
%%  \begin{describe}
%  {\ttfamily stream = open(path,mode)}
%    Return a Python \texttt{file} object for reading or writing the
%    file located at \texttt{path}.  Mode is one of '\texttt{r}',
%    '\texttt{w}' or '\texttt{a}' for reading, writing (truncates on open), appending.
%    You can add a `\texttt{+}' character to enable read+write (other
%    effects being the same).
%%  \end{describe}
%
%%  \begin{describe}
%  {\ttfamily \emph{stream}.close()}
%    Close an open file.
%%  \end{describe}
%
%%  \begin{describe}
%  {\ttfamily \textbf{for} line \textbf{in} stream:}
%    Loop over lines in the file one by one.
%%  \end{describe}
%
%%  \begin{references}
%%    \url{http://docs.python.org/library/stdtypes.html#file-objects}
%%  \end{references}
%\end{frame}
%
%%%%%%%%%%%%%%%%%%%%%%%%%%%%%%%%%%%%%%%%%%%%%%%%%%%%%%%%%%%%%%%%%%%%%%%%%%%%%%%%%%%%
%\begin{frame}[fragile]\frametitle{File I/O, II}
%
%  The \lstinline|read(n)| method can be used to read \emph{at most}
%  \lstinline|n| bytes from a file-like object:
%\begin{lstlisting}
%>>> s = stream.read(2)
%>>> print(s)
%'py'
%\end{lstlisting}
%  If \lstinline|n| is omitted, \texttt{read()} reads until end-of-file.
%
%%  \begin{references}
%%    \url{http://docs.python.org/library/stdtypes.html#file-objects}
%%  \end{references}
%\end{frame}


%%%%%%%%%%%%%%%%%%%%%%%%%%%%%%%%%%%%%%%%%%%%%%%%%%%%%%%%%%%%%%%%%%%%%%%%%%%%%%%%%%%
\begin{frame}[fragile]\frametitle{Filesystem operations, I}
\texttt{os} module functions:

  \begin{lstlisting}
  os.getcwd(),   os.chdir(path)
  \end{lstlisting}
   Get current working directory. Change to \texttt{path}.
  \begin{lstlisting}
  os.listdir(dir)
  \end{lstlisting}
    Return list of entries in directory \texttt{dir} (omitting
    `\texttt{.}' and `\texttt{..}')
  \begin{lstlisting}
  os.mkdir(path)
  \end{lstlisting}
    Create a directory; fails if the directory already exists.
    Assumes that all parent directories exist already.
  \begin{lstlisting}
  os.makedirs(path)
\end{lstlisting}
Create a directory; no-op if the directory already exists.
Creates all the intermediate-level directories needed to contain the leaf.
%\begin{lstlisting}
%  os.rename(old,new)
%\end{lstlisting}
%    Rename a file or directory from \texttt{old} to \texttt{new}.
\end{frame}


%%%%%%%%%%%%%%%%%%%%%%%%%%%%%%%%%%%%%%%%%%%%%%%%%%%%%%%%%%%%%%%%%%%%%%%%%%%%%%%%%%%
\begin{frame}[fragile]\frametitle{Filesystem operations, II}
  These functions are available from the \texttt{os.path} module.

\begin{lstlisting}
os.path.exists(path), os.path.isdir(path), os.path.isfile(path)
\end{lstlisting}
    Return \texttt{True} if \texttt{path} exists / is a directory / is
    a regular file.
  \begin{lstlisting}
os.path.basename(path),
os.path.dirname(path)
  \end{lstlisting}
    Return the base name (the part after the last `\texttt{/}'
    character) or the directory name (the part before the last
    \texttt{/} character).
  \begin{lstlisting}
  os.path.abspath(path)
  \end{lstlisting}
    Make \texttt{path} absolute (i.e., start with a \texttt{/}).
\end{frame}

%%%%%%%%%%%%%%%%%%%%%%%%%%%%%%%%%%%%%%%%%%%%%%%%%%%%%%%%%%%%%%%%%%%%%%%%%%%%%%%%%%%%
%\begin{frame}[fragile]\frametitle{Exercises}
%Write a function \lstinline|cat(filename)| that prints the whole contents of a file. Test it with the https://raw.github.com/gc3-uzh-ch/python-course/master/welcome.py file:
%\begin{lstlisting}
%>>> cat('welcome.py')
%#! /usr/bin/env python
%
%print ("Welcome to Python!")
%\end{lstlisting}
%Write a function \lstinline|load_data(filename)| that reads a file containing one integer number per line, and return a list of the integer values. Test it with the https://raw.github.com/gc3-uzh-ch/python-course/master/values.dat  file:
%\begin{lstlisting}
%>>> load_data('values.dat')
%[299850, 299740, 299900, 300070, 299930]
%\end{lstlisting}
%\end{frame}

%%%%%%%%%%%%%%%%%%%%%%%%%%%%%%%%%%%%%%%%%%%%%%%%%%%%%%%%%%%%%%%%%%%%%%%%%%%%%%%%%%%
\begin{frame}[fragile]\frametitle{Exercises}
 \begin{itemize}
    \item Write a program that reads the https://raw.github.com/gc3-uzh-ch/python-course/master/euro.csv
    file and populates a dictionary from it: currency names (first column) are the dictionary keys, conversion rates (second column) are the dictionary values.
%    \item Write a function that reads a file and returns its content as a list of strings (one string per line). Read the file with filename data/restaurant-names.txt. If you stored your notebook under Student\_Notebooks the full filename is /data/restaurant-names.txt
\end{itemize}

\end{frame}

%%%%%%%%%%%%%%%%%%%%%%%%%%%%%%%%%%%%%%%%%%%%%%%%%%%%%%%%%%%%%%%%%%%%%%%%%%%%%%%%%%%
\begin{frame}[fragile]\frametitle{Exercises}
 \begin{itemize}
    \item Write a function that reads the n-th column of a CSV file and returns its contents. (Reuse the function that you wrote above.) Then reads the file data/baseball.csv and return the content of the 5th column (team).
    \item The command below will create a file called phonetest.txt. Write code that:
    Reads the file phonetest.txt;
    Write a function that takes as input a string, and removes any non-digit characters;
    Print out the "clean" string, without any non-digit characters;
\end{itemize}
\end{frame}

%%%%%%%%%%%%%%%%%%%%%%%%%%%%%%%%%%%%%%%%%%%%%%%%%%%%%%%%%%%%%%%%%%%%%%%%%%%%%%%%%%%
\begin{frame}[fragile]\frametitle{Exercises}
\begin{itemize}
\item
    Write a function \lstinline|wordcount(filename)| that reads a text file and returns a dictionary, mapping words into occurrences (disregarding case) of that word in the text.  
    Test it with the https://raw.github.com/gc3-uzh-ch/python-course/master/lorem\_ipsum.txt file:
    \begin{lstlisting}
>>> wordcount('lorem_ipsum.txt')
{'and': 3, 'model': 1, 'more-or-less': 1,
 'letters': 1, ...
    \end{lstlisting}
%\item
%     For the purposes of this exercise, a ``word'' is defined as a sequence of letters and the character ``-'', i.e., ``e-mail'' and ``more-or-less'' should both be counted as a single word.
%\item
%     You might want to have a look at the http://docs.python.org/2/library/string.html module, for pre-defined sets of alphabetic and punctuation characters.
 \end{itemize}
\end{frame}



%%%%%%%%%%%%%%%%%%%%%%%%%%%%%%%%%%%%%%%%%%%%%%%%%%%%%%%%%%%%%%%%%%%%%%%%%%%%%%%%%%%
\begin{frame}[fragile]\frametitle{Files Operations (recap)}
  \begin{itemize}
  \item input \lstinline{input = open('data', 'r')}
  \item read all \lstinline{S = input.read()}
  \item read N bytes \lstinline{S = input.read(N)}
  \item read next \lstinline{S = input.readline()}
  \item read in lists \lstinline{L = input.readlines()}
  \item output \lstinline{output = open('/tmp/spam', 'w')}
  \item write \lstinline{output.write(S)}
  \item write \lstinline{strings output.writelines(L)}
  \item close \lstinline{output.close()}
  \end{itemize}
\end{frame}

%
%%%%%%%%%%%%%%%%%%%%%%%%%%%%%%%%%%%%%%%%%%%%%%%%%%%%%%%%%%%%%%%%%%%%%%%%%%%%%%%%%%%%
%\begin{frame}[fragile]\frametitle{Command line arguments}
%  The \texttt{sys} module provides access to some variables used or
%  maintained by the interpreter.
%
%  One of such variables is a list containing the arguments passed on
%  the command line.
%
%  
%  \textbf{Example:} This is a simple script that
%  prints the command line arguments used to invoke it:
%
%  \begin{lstlisting}
%import sys
%print(sys.argv)
%  \end{lstlisting}
%
%
%Calling the script as:
%\begin{lstlisting}
%$ python script.py foo bar
%\end{lstlisting}
%yields the following result:
%\begin{lstlisting}
%['script.py', 'foo', 'bar']
%\end{lstlisting}
%\end{frame}
%
%
%%%%%%%%%%%%%%%%%%%%%%%%%%%%%%%%%%%%%%%%%%%%%%%%%%%%%%%%%%%%%%%%%%%%%%%%%%%%%%%%%%%%
%\begin{frame}[fragile]  \frametitle{(Homework)}
%\begin{itemize}
%\item
%    Write a Python program \texttt{rename.py} with the following
%    command-line:
%\begin{lstlisting}[language=sh]
%python rename.py EXT1 EXT2 DIR [DIR ...]
%\end{lstlisting}
%\item
%   \begin{itemize}
%    \item {\bf ext1,ext2} Are file name extensions (without the leading dot), e.g., \texttt{jpg} and \texttt{jpeg}.
%    \item {\bf dir} Is a directory path; possibly, many directories names can
%      be given on the command-line.
%    \end{itemize}
%\item
%    The \texttt{rename.py} command should rename all files in
%    directory DIR, that end with extension \texttt{ext1} to end with
%    extension \texttt{ext2} instead.
%\end{itemize}
%\end{frame}