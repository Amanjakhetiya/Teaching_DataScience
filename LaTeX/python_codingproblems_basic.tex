%%%%%%%%%%%%%%%%%%%%%%%%%%%%%%%%%%%%%%%%%%%%%%%%%%%%%%%%%%%%%%%%%%%%%%%%%%%%%%%%%%
\begin{frame}[fragile]\frametitle{}
\begin{center}
{\Large Coding Problems : Basic}
\end{center}

{\tiny (Ref: How to Win Coding Competitions: Secrets of Champions - EdX)}
\end{frame}

%%%%%%%%%%%%%%%%%%%%%%%%%%%%%%%%%%%%%%%%%%%%%%%%%%%%%%%%%%%%%%%%%%%%%%%%%%%%%%%%%%%
\begin{frame}[fragile]\frametitle{Setup}
\begin{itemize}
\item Single python source file
\item Have both input and output files in the current directory of your program.
\item The input file will exist and will be available for reading.
\item The output.txt needs to be generated with format described in the problem statement.
\end{itemize}
\end{frame}

%%%%%%%%%%%%%%%%%%%%%%%%%%%%%%%%%%%%%%%%%%%%%%%%%%%%%%%%%%%%%%%%%%%%%%%%%%%%%%%%%%%
\begin{frame}[fragile]\frametitle{Test Setup}
\begin{itemize}
\item Given two integer numbers, $A$  and $B$, output $A+B$.
\item Input: The input file contains one line with two integer numbers, $A$ and $B$, $(-10^9 \le A,B \le 10^9)$ separated by one whitespace.
\item Output: Output $A+B$  in the first and only line of the output file.
\item Example:

\begin{table}[h!]
    \begin{tabular}{|l|r|}
	\hline
      input.txt & ouput.txt\\
      \hline
      23 11 & 34\\
	  \hline
    \end{tabular}
\end{table}
\end{itemize}
\end{frame}

%%%%%%%%%%%%%%%%%%%%%%%%%%%%%%%%%%%%%%%%%%%%%%%%%%%%%%%%%%%%%%%%%%%%%%%%%%%%%%%%%%%
\begin{frame}[fragile]\frametitle{Test Setup: My Solution}
\begin{lstlisting}
a = 0
b = 0
with open('input.txt','r') as infile:
	arr = infile.read().split()
	a = int(arr[0])
	b = int(arr[1])
	
c = a + b

with open('output.txt','w') as outfile:
	outfile.write(str(c))
	outfile.write('\n')
\end{lstlisting}
\end{frame}

%%%%%%%%%%%%%%%%%%%%%%%%%%%%%%%%%%%%%%%%%%%%%%%%%%%%%%%%%%%%%%%%%%%%%%%%%%%%%%%%%%%
\begin{frame}[fragile]\frametitle{Warmup}
\begin{itemize}
\item Given two integer numbers, $A$  and $B$, output $A+B^2$.
\item Input: The input file contains one line with two integer numbers, $A$ and $B$, $(-10^9 \le A,B \le 10^9)$ separated by one whitespace.
\item Output: Output $A+B^2$  in the first and only line of the output file.
\item Example:

\begin{table}[h!]
    \begin{tabular}{|l|r|}
	\hline
      input.txt & ouput.txt\\
      \hline
      23 11 & 144\\
	  \hline
    \end{tabular}
\end{table}
\end{itemize}
\end{frame}

%%%%%%%%%%%%%%%%%%%%%%%%%%%%%%%%%%%%%%%%%%%%%%%%%%%%%%%%%%%%%%%%%%%%%%%%%%%%%%%%%%%
\begin{frame}[fragile]\frametitle{Warmup: My Solution}
\begin{lstlisting}
a = 0
b = 0
with open('input.txt','r') as infile:
	arr = infile.read().split()
	a = int(arr[0])
	b = int(arr[1])
	
c = a + b**2

with open('output.txt','w') as outfile:
	outfile.write(str(c))
	outfile.write('\n')
\end{lstlisting}
\end{frame}

%%%%%%%%%%%%%%%%%%%%%%%%%%%%%%%%%%%%%%%%%%%%%%%%%%%%%%%%%%%%%%%%%%%%%%%%%%%%%%%%%%%
\begin{frame}[fragile]\frametitle{Three-bonacci}
\begin{itemize}
\item This problem is about a funny extension of the well-known Fibonacci numbers, called Three-bonacci numbers. We define them as follows:
\begin{equation}
T_i = \begin{cases}
      	A_0 & \text{ if } i = 0; \\
        A_1 & \text{ if } i = 1; \\
        A_2 & \text{ if } i = 2; \\
        T_{i - 1} + T_{i - 2} - T_{i - 3} & \text{ otherwise.}
      \end{cases}
\end{equation}
We give you $A_0$,$A_1$,$A_2$, and $n$. Your task is to calculate $T_n$.
\item Input: Four integer numbers, $A_0$,$A_1$,$A_2$, and $n$
\item Output: Output $T_n$  in the first and only line of the output file.
\item Example:

\begin{table}[h!]
    \begin{tabular}{|l|r|}
	\hline
      input.txt & ouput.txt\\
      \hline
      1 2 3 4 & 5\\
	  \hline
    \end{tabular}
\end{table}
\end{itemize}
\end{frame}

%%%%%%%%%%%%%%%%%%%%%%%%%%%%%%%%%%%%%%%%%%%%%%%%%%%%%%%%%%%%%%%%%%%%%%%%%%%%%%%%%%%
\begin{frame}[fragile]\frametitle{Three-bonacci: My Solution}
\begin{lstlisting}
def getTValue(index):
	if index < 3:
		return arr[index]
	else:
		return getTValue(index -1) + getTValue(index -2) - getTValue(index -3)
		
arr = []
with open('input.txt','r') as infile:
	arr = [int(x) for x in infile.read().split()]
n = arr.pop()
c = getTValue(n)

with open('output.txt','w') as outfile:
	outfile.write(str(c))
	outfile.write('\n')
\end{lstlisting}
\end{frame}

%%%%%%%%%%%%%%%%%%%%%%%%%%%%%%%%%%%%%%%%%%%%%%%%%%%%%%%%%%%%%%%%%%%%%%%%%%%%%%%%%%%
\begin{frame}[fragile]\frametitle{Prepare Yourself to Competitions!}
\begin{itemize}
\item  Jamie understood that there are two ways to improve his skills: studying theory and practicing a lot. There are $n$ days before the next programming competition. 
\item Jamie determined two numbers for each of these days: $t_i$ is how much his ability to solve problems will improve if he studies theory at the $i$-th day, and $p_i$ is how much it will improve if he practices a lot at the $i$-th day. 
\item Every day should be entirely dedicated to either theory or practice. 
\item Additionally, at least one of these days should be theoretical, and at least one should be practical.
\item Input: 1st line: $n$, 2nd line: $p_i$s, 3rd line : $t_i$s
\item Output: Output the maximum possible value of ability to solve problems, which Jamie can achieve in $n$ days. 
\end{itemize}
\end{frame}

%%%%%%%%%%%%%%%%%%%%%%%%%%%%%%%%%%%%%%%%%%%%%%%%%%%%%%%%%%%%%%%%%%%%%%%%%%%%%%%%%%%
\begin{frame}[fragile]\frametitle{Prepare Yourself to Competitions!}

\begin{table}[h!]
    \begin{tabular}{|l|r|}
	\hline
      input.txt & ouput.txt\\
      \hline
		\makecell[l]{
		2 \\
		1 2 \\ 
		2 1} & 4\\ \hline
		\makecell[l]{
		3\\
		1 2 3\\
		1 2 3} & 6\\		
	  \hline
    \end{tabular}
\end{table}
\end{frame}


%%%%%%%%%%%%%%%%%%%%%%%%%%%%%%%%%%%%%%%%%%%%%%%%%%%%%%%%%%%%%%%%%%%%%%%%%%%%%%%%%%%
\begin{frame}[fragile]\frametitle{Prepare Yourself to Competitions!: My Solution}
\begin{lstlisting}
input_lines = []
with open('input.txt','r') as infile:
	input_lines = infile.readlines()
n = int(input_lines[0])
p_arr = [int(x) for x in input_lines[1].split()]
t_arr = [int(x) for x in input_lines[2].split()]

beenInP = -1
beenInT = -1
c = 0
for i in range(n):
	if p_arr[i] > t_arr[i]:
		beenInP = 1
		c += p_arr[i]
	elif t_arr[i] > p_arr[i]:
		c += t_arr[i]
		beenInT = 1
	else:
		c += t_arr[i]
		beenInP = 1
		beenInT = 1	
\end{lstlisting}
\end{frame}

%%%%%%%%%%%%%%%%%%%%%%%%%%%%%%%%%%%%%%%%%%%%%%%%%%%%%%%%%%%%%%%%%%%%%%%%%%%%%%%%%%%
\begin{frame}[fragile]\frametitle{Prepare Yourself to Competitions!: My Solution}
\begin{lstlisting}
if beenInP != 1 and beenInT != 1:
	c = 0
	
with open('output.txt','w') as outfile:
	outfile.write(str(c))
	outfile.write('\n')
\end{lstlisting}
\end{frame}

%%%%%%%%%%%%%%%%%%%%%%%%%%%%%%%%%%%%%%%%%%%%%%%%%%%%%%%%%%%%%%%%%%%%%%%%%%%%%%%%%%%
\begin{frame}[fragile]\frametitle{Create a Team!}
\begin{itemize}
\item  Andrew decided to create a team to participate in programming competitions. 
\item He invited his classmates, Peter and Paul. Now they have a question of role distribution. 
\item Andrew thinks that there are three roles in the team: a ``coder'', a ``mathematician'', and a ``tester''. 
\item Now, they have a $3x3$ table, where the first row is for Andrew, the second row is for Peter, and the third one is for Paul. 
\item The columns correspond to the roles: first to a ``coder'', second to a ``mathematician'', third to a ``tester''. 
\item As an example, the number in the third column of the second row shows how good is Peter as a ``tester''.
\end{itemize}
\end{frame}

%%%%%%%%%%%%%%%%%%%%%%%%%%%%%%%%%%%%%%%%%%%%%%%%%%%%%%%%%%%%%%%%%%%%%%%%%%%%%%%%%%%
\begin{frame}[fragile]\frametitle{Create a Team!}
\begin{itemize}
\item  How the guys want to distribute the roles in such a way that the team performs in a most efficient way. Of course, every person can take exactly one role, and every role should be occupied by exactly one person. The efficiency of the assignment where Andrew performs with the efficiency of $A$ , Peter performs with the efficiency of $B$ and Paul with $C$, is equal to $\sqrt{A^2 + B^2 + C^2}$.
\item Help Andrew, Peter and Paul to find the role distribution with the maximum efficiency.
\item Input: The input file contains three lines, each containing three integer numbers 
\item Output: Output one number – the maximum efficiency
\end{itemize}
\end{frame}
%%%%%%%%%%%%%%%%%%%%%%%%%%%%%%%%%%%%%%%%%%%%%%%%%%%%%%%%%%%%%%%%%%%%%%%%%%%%%%%%%%%
\begin{frame}[fragile]\frametitle{Prepare Yourself to Competitions!}

\begin{table}[h!]
    \begin{tabular}{|l|r|}
	\hline
      input.txt & ouput.txt\\
      \hline
		\makecell[l]{
		1 1 1\\
		1 1 1\\
		1 1 1} & 1.7320\\ \hline
		\makecell[l]{
		1 2 3\\
		6 5 4\\
		7 8 9} & 11.0\\		
	  \hline
    \end{tabular}
\end{table}
\end{frame}


%%%%%%%%%%%%%%%%%%%%%%%%%%%%%%%%%%%%%%%%%%%%%%%%%%%%%%%%%%%%%%%%%%%%%%%%%%%%%%%%%%%
\begin{frame}[fragile]\frametitle{Prepare Yourself to Competitions!: My Solution}
\begin{lstlisting}
m = []
with open('input.txt','r') as infile:
	for line in infile:
		row = [int(x) for x in line.split()]
		m.append(row)
		
def getDistance(a,b,c):
	return (a**2+b**2+c**2)**0.5

v1 = getDistance(m[0][0],m[1][2],m[2][1])
v2 = getDistance(m[0][0],m[1][1],m[2][2])
v3 = getDistance(m[0][1],m[1][0],m[2][2])
v4 = getDistance(m[0][1],m[1][2],m[2][0])
v5 = getDistance(m[0][2],m[1][1],m[2][0])
v6 = getDistance(m[0][2],m[1][0],m[2][1])

c = max(v1,v2,v3,v4,v5,v6)

with open('output.txt','w') as outfile:
	outfile.write(str(c))
	outfile.write('\n')
\end{lstlisting}
% Very crude. I don't like it. Need to good matrix permutations logic.
\end{frame}
