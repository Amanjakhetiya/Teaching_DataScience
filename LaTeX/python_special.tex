%%%%%%%%%%%%%%%%%%%%%%%%%%%%%%%%%%%%%%%%%%%%%%%%%%%%%%%%%%%%%%%%%%%%%%%%%%%%%%%%%%
\begin{frame}[fragile]\frametitle{}
\begin{center}
{\Large Special Methods}
\end{center}
\end{frame}

%%%%%%%%%%%%%%%%%%%%%%%%%%%%%%%%%%%%%%%%%%%%%%%%%%%%%%%%%%%%%%%%%%%%%%%%%%%%%%%%%%%
\begin{frame}[fragile]\frametitle{What we shall see in this part}

  How to define custom behavior for Python's standard operators and
  functions on user-defined objects.

  
  (Technically called ``operator overloading.'')
\end{frame}

%%%%%%%%%%%%%%%%%%%%%%%%%%%%%%%%%%%%%%%%%%%%%%%%%%%%%%%%%%%%%%%%%%%%%%%%%%%%%%%%%%%
\begin{frame}[fragile]\frametitle{Python's special methods}

\begin{itemize}
\item Names that start and end with two underscores e.g., \_\_init\_\_ have a special significance in Python.
\item However, there is nothing special in the way we define those
  methods: the syntax is the same as for any other method.
  \item    Most of the special methods directly map to Python's operators (e.g.,
  ``\texttt{+}'', ``\texttt{==}'' or ``\texttt{in}'').
\end{itemize}
\end{frame}

%%%%%%%%%%%%%%%%%%%%%%%%%%%%%%%%%%%%%%%%%%%%%%%%%%%%%%%%%%%%%%%%%%%%%%%%%%%%%%%%%%%
\begin{frame}[fragile]\frametitle{First Example (1/2)}
 Let us rename the \texttt{show} method to \texttt{\_\_str\_\_}.
\begin{lstlisting}
class Vector(object):
  """A 2D Vector."""
  def __init__(self, x, y):
    self.x = x
    self.y = y
    
  def add(self, other):
    return Vector(self.x+other.x,
                  self.y+other.y)
                  
  def mul(self, scalar):
    return Vector(scalar*self.x, scalar*self.y)
    
  def __str__(self):
    return ("<%g,%g>" % (self.x, self.y))
\end{lstlisting}
  
\end{frame}

%%%%%%%%%%%%%%%%%%%%%%%%%%%%%%%%%%%%%%%%%%%%%%%%%%%%%%%%%%%%%%%%%%%%%%%%%%%%%%%%%%%
\begin{frame}[fragile]\frametitle{First Example (2/2)}
\begin{lstlisting}
>>> from vector2 import Vector
>>> v = Vector(0,1)
>>> print(v)
<0,1>
\end{lstlisting}
\begin{itemize}
\item Now Python's built-in \texttt{print} behaves like the \texttt{show} method did!
\item Actually, \texttt{print} uses Python's built-in function
  \texttt{str()} to convert an object to a string, and then prints
  this string.
\item By defining the \texttt{\_\_str\_\_} method, we
    override the default behavior of Python's \texttt{str()} for
    objects of class \texttt{Vector}.
\end{itemize}
\end{frame}

%%%%%%%%%%%%%%%%%%%%%%%%%%%%%%%%%%%%%%%%%%%%%%%%%%%%%%%%%%%%%%%%%%%%%%%%%%%%%%%%%%%
\begin{frame}[fragile]\frametitle{Second Example: equality testing (1/3)}

\begin{itemize}
\item  Can we test two instanced of class \texttt{Vector} for equality?
\begin{lstlisting}
>>> from vector2 import Vector
>>> v1 = Vector(0,1)
>>> v2 = Vector(0,1)
>>> v1 == v2
False
\end{lstlisting}

\item  Python does not know how to test if two user defined objects are  equal.

\item  By default ``\texttt{==}'' behaves like the ``\texttt{is}'' operator on user-defined classes
\item Two
  user-defined objects are considered equal if and only if they are
  the same object.

\item  This can be changed by adding a \texttt{\_\_eq\_\_} method.
\end{itemize}
\end{frame}

%%%%%%%%%%%%%%%%%%%%%%%%%%%%%%%%%%%%%%%%%%%%%%%%%%%%%%%%%%%%%%%%%%%%%%%%%%%%%%%%%%%
\begin{frame}[fragile]\frametitle{Equality \emph{vs} identity}
  
  \begin{itemize}
\item  The \texttt{is} operator returns \texttt{True} if two names refer to
  the same instance; the \texttt{==} operator compares the
  \emph{values} of two objects.\footnote{A class can define how
    exactly the \texttt{==} operator should carry out the comparison.}

  \item  
  Note that two instances may be equal in any respect yet be
  different instances: \emph{equality is not identity!}
\begin{lstlisting}
>>> dt4 = date(2012,9,28)
>>> dt5 = date(2012,9,28)
>>> dt4 == dt5
True
>>> dt4 is dt5
False
\end{lstlisting}
\end{itemize}
\end{frame}

%%%%%%%%%%%%%%%%%%%%%%%%%%%%%%%%%%%%%%%%%%%%%%%%%%%%%%%%%%%%%%%%%%%%%%%%%%%%%%%%%%%
\begin{frame}[fragile]\frametitle{Second Example: equality testing (2/3)}
      Let us add an eq method.
\begin{lstlisting}
class Vector(object):
  """A 2D Vector."""
  def __init__(self, x, y):
    self.x = x
    self.y = y
  # (code omitted)
  def __str__(self):
    return ("<%g,%g>" % (self.x, self.y))
  \textbf{def} __eq__(\textbf{self}, other):
    return (self.x == other.x) and (self.y == other.y)
\end{lstlisting}
  

\end{frame}

%%%%%%%%%%%%%%%%%%%%%%%%%%%%%%%%%%%%%%%%%%%%%%%%%%%%%%%%%%%%%%%%%%%%%%%%%%%%%%%%%%%
\begin{frame}[fragile]\frametitle{Second Example: equality testing (3/3)}
\begin{lstlisting}
>>> from vector3 import Vector
>>> v1 = Vector(0,1)
>>> v2 = Vector(0,1)
>>> v1 == v2
True
\end{lstlisting}

   {\bfseries By defining the eq method, we
    define the behavior of Python's equality test == for
    objects of class \texttt{Vector}.}
\end{frame}

%%%%%%%%%%%%%%%%%%%%%%%%%%%%%%%%%%%%%%%%%%%%%%%%%%%%%%%%%%%%%%%%%%%%%%%%%%%%%%%%%%%
\begin{frame}[fragile]\frametitle{3rd Example: vector addition (1/3)}
  

      The add special method defines the behavior of the
      ``\texttt{+}'' operator.

      
      Let's just rename \texttt{add} \\ to \texttt{\_\_add\_\_}.
\begin{lstlisting}
class Vector(object):
  """A 2D Vector."""
  def __init__(self, x, y):
    self.x = x
    self.y = y
  def __str__(self):
    return ("<%g,%g>" % (self.x, self.y))
  def __eq__(self, other):
    return (self.x == other.x) and (self.y == other.y)
  \textbf{def} __add__(\textbf{self}, other):
    return Vector(self.x+other.x, self.y+other.y)
\end{lstlisting}

\end{frame}

%%%%%%%%%%%%%%%%%%%%%%%%%%%%%%%%%%%%%%%%%%%%%%%%%%%%%%%%%%%%%%%%%%%%%%%%%%%%%%%%%%%
\begin{frame}[fragile]\frametitle{3rd Example: vector addition (2/3)}

  Now vector addition works with the usual ``\texttt{+}'' operator:
\begin{lstlisting}
>>> from vector4 import Vector
>>> v1 = Vector(1,0)
>>> v2 = Vector(0,1)
>>> print (v1 + v2)
<1,1>
\end{lstlisting}
\end{frame}

%%%%%%%%%%%%%%%%%%%%%%%%%%%%%%%%%%%%%%%%%%%%%%%%%%%%%%%%%%%%%%%%%%%%%%%%%%%%%%%%%%%
\begin{frame}[fragile]\frametitle{3rd Example: vector addition (3/3)}

  What if we add inhomogeneous objects, e.g., a vector and a number?
\begin{lstlisting}
>>> from vector4 import Vector
>>> v1 = Vector(1,0)
>>> print (v1 + 5.0)
Traceback (most recent call last):
  File "<stdin>", line 1, in <module>
  File "vector4.py", line 14, in __add__
    return Vector(self.x+other.x, self.y+other.y)
AttributeError: 'float' object has no attribute 'x'
\end{lstlisting}

   In this case, an error is the correct behavior: a vector can only
  be summed to another vector.

   We shall see in the next part that sometimes it makes sense to
  allow in homogeneous operations, and how to implement them.
\end{frame}

%%%%%%%%%%%%%%%%%%%%%%%%%%%%%%%%%%%%%%%%%%%%%%%%%%%%%%%%%%%%%%%%%%%%%%%%%%%%%%%%%%%
\begin{frame}[fragile]\frametitle{4th Example: vector multiplication (1/3)}
      The mul special method defines the behavior of
      the ``\texttt{*}'' operator.
\begin{lstlisting}
class Vector(object):
  """A 2D Vector."""
  def __init__(self, x, y):
    self.x = x
    self.y = y
  def __str__(self):
    return ("<%g,%g>" % (self.x, self.y))
  def __eq__(self, other):
    return (self.x == other.x) and (self.y == other.y)
  def __add__(self, other):
    return Vector(self.x+other.x, self.y+other.y)
  \textbf{def} __mul__(\textbf{self}, scalar):
    return Vector(scalar*self.x, scalar*self.y)
\end{lstlisting}
\end{frame}

%%%%%%%%%%%%%%%%%%%%%%%%%%%%%%%%%%%%%%%%%%%%%%%%%%%%%%%%%%%%%%%%%%%%%%%%%%%%%%%%%%%
\begin{frame}[fragile]\frametitle{4th Example: vector multiplication (2/3)}
  Now vector multiplication works with the usual ``\texttt{*}'' operator:
\begin{lstlisting}
>>> from vector5 import Vector
>>> v1 = Vector(1,2)
>>> v1 * 3 == Vector(3, 6)
>>> print (v1 * 2)
<2,4>
\end{lstlisting}
\end{frame}

%%%%%%%%%%%%%%%%%%%%%%%%%%%%%%%%%%%%%%%%%%%%%%%%%%%%%%%%%%%%%%%%%%%%%%%%%%%%%%%%%%%
\begin{frame}[fragile]\frametitle{4th Example: vector multiplication (3/3)}
\begin{itemize}
\item   

  Note that:
\begin{lstlisting}[basicstyle=\ttfamily\scriptsize]
>>> from vector5 import Vector
>>> v1 = Vector(1,2)
>>> 3 * v1
Traceback (most recent call last):
  File "<stdin>", line 1, in <module>
TypeError: unsupported operand type(s) for *: 'int' and 'Vector'
\end{lstlisting}

\item  Order matters! Our \texttt{\_\_mul\_\_} method requires a \texttt{Vector} instance first, and a number second.

\item  The operation with swapped operands is called \texttt{\_\_rmul\_\_}, but treating this in detail would take us too far!

\item Take-home message: \textbf{Operations defined with special
    methods are not automatically commutative, transitive or any other
    property you normally associate with the operators \texttt{+},
    \texttt{*}, etc.}
\end{itemize}    
\end{frame}

%%%%%%%%%%%%%%%%%%%%%%%%%%%%%%%%%%%%%%%%%%%%%%%%%%%%%%%%%%%%%%%%%%%%%%%%%%%%%%%%%%%
\begin{frame}[fragile]{Further reading}

\begin{itemize}
\item   A good and readable introduction to Python's special methods:
  \begin{center}
    \url{http://www.rafekettler.com/magicmethods.html}
  \end{center}
\item Special methods hook directly into Python's syntax.  They can
  enhance readability or make code completely obscure.  Use them
  sparingly, and remember: \emph{with great power comes great
    responsibility!}
\end{itemize}


  
  
\end{frame}
